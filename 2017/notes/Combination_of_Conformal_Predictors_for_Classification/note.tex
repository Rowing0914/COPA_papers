\documentclass[10pt,a4paper]{article}
\usepackage[utf8]{inputenc}
\usepackage{amsmath}
\usepackage{amsfonts}
\usepackage{amssymb}
\usepackage{algorithm2e}
\usepackage{graphicx}
\graphicspath{ {./images/} }
\usepackage{biblatex}
\addbibresource{sample.bib}
\author{Norio Kosaka}
\title{Review: Combination of Conformal Predictors for Classification}
\begin{document}

\maketitle

\section{Paper Profile}
\begin{itemize}
\item Title: Combination of Conformal Predictors for Classification
\item Author: Paolo Toccaceli, Alexander Gammerman
\item Organisation: Royal Holloway, University of London, CS Research Centre
\item Publish Year: 2017
\item URL: http://proceedings.mlr.press/v60/yanovich17a/yanovich17a.pdf
\end{itemize}

\section{Prerequisites}
\begin{itemize}
    \item Conformal Prediction: Intuitively it is the methods that uses past experience to determine precise level of confidence in new predictions, and it employs statistical indicators to evaluate the validity of the prediction
    Good PPT: http://www.clrc.rhul.ac.uk/copa2017/presentations/CP{\_}Tutorial{\_}2017.pdf
\end{itemize}

\section{Contents in the paper}
\begin{enumerate}
\item Introduction
\item Results Description
    \begin{enumerate}
        \item Manifold Learning Data Model
        \item Statistics Form
        \item Main Results
    \end{enumerate}
\item Data Model
\item Main Results
\item Proof of Main Theorems
\item Conclusion
\end{enumerate}

\section{Abstract}
The author proposed some possible approaches to the combination of \textit{Conformal Predictors} in the binary classification case.
\begin{itemize}
    \item p-value combination techniques
    \item Calibration of p-values into Bayes factors
\end{itemize}
And the result shows that the \textbf{P-value combined with Fisher's Method} worked fine, when ranking compounds by strength of evidence.

\section{Introduction}
\textbf{Conformal Predictors}(CP)(Vovk et al., 2005 \cite{vovk2005conformal}; Gammerman and Vovk, 2007\cite{gammerman2007hedging}) was proposed to provide a validity property on the prediction stage.The efficiency of CP hugely relies on underlying \textit{Machine Learning algorithms}. The objective of this paper is to explore ways to improve CP by some forms \textit{ensembling}, which in general refers to the approaches, for example, Random Forests or Bagging. But the author claimed that he has differentiate the proposition from other approaches in a way that it does not explicitly aim at combating overfitting and correlation per se. So, the challenge which he has encountered was to find a method of wide applicability that combines the predictions in a synergistic way.

\section{Conformal Predictors}
Key points:
\begin{itemize}
    \item the training set is made up of $l$ independent identically distributed samples
    \item CP assigns a p-value to a prediction $y$
    \item \textbf{Non-Conformity Measure(NCM)} is a real-valued function $\alpha$ expressing how odd the sampled example is among the training set.
    \item Applying one of the following methods to a test set
    \begin{itemize}
        \item Given a significance level $\epsilon$, a \textit{region predictor} outputs for each test object the set of labels, such that the actual label is not in the set no more than a fraction $\epsilon$ of the times.
        \item Or, one can pick the largest p-value for a given test object alongside with its credibility and confidence.
    \end{itemize}
\end{itemize}
Two forms of CP
\begin{itemize}
    \item Transductive CP: computationally expensive, requires the computation from scratch for each object
    \item Inductive CP: requires just one training of the underlying, yet also it requires that the training data set which is split into a proper training set and a calibration set.
\end{itemize}

\section{Requirements for CP combination}
The research of the problem of combining p-values to obtain a single test for a common hypothesis has a long history, originating to the paper from Fiher's work \cite{fisher1992statistical}, yet there has not been identified the general enough approach to apply to any kind of real-world problem. Indeed, the combination of p-values from different \textit{CP} on the same test object is a variant of the approaches above and it has aims described below.
\begin{itemize}
    \item \textbf{Preserve Validity}: for the output of the combination method to be a CP, this is a necessary property.
    \item \textbf{Improve Efficiency}: smaller prediction sets must result from a desirable method of combination.
\end{itemize}

There are two principle in \textit{Mondrian Inductive CP}\\
(http://onlineprediction.net/?n=Main.MondrianConformalPredictor).
 \begin{enumerate}
     \item The p-values from the same CP for the various test objects do not necessarily follow the uniform distribution.
     \item The p-values from different CP for the same test object are not independent.
 \end{enumerate}

\section{Methods from Statistical Hypothesis Testing}
In general there are two types of p-value combination methods.
\begin{itemize}
    \item Order-statistic methods by Davidov(2011)\cite{davidov2011combining}
    \item Quantile method, here we consider two methods more: Fisher's method and Stouffer's method
\end{itemize}

\subsection{Comparison of Fisher's method(Chi-square method) and Soutffer's method (z-transform method)}
"Fisher’s method is asymptotically optimal among essentially all methods of combining
independent tests" from Littel and Folks(1971) \cite{littell1971asymptotic}

\section{Calibration to Bayes Factors}
P-values can be transformed into \textit{Bayes Factors}, which defined as 

\begin{gather}
B_{\theta}(x) = \frac{L_{x}(\theta)}{\int_{\Theta} L_{x}(\theta) dQ(\theta)}
\end{gather}
where $L_{x}(\theta)$ is the likelihood of $x$ given $\theta$ and $Q(\theta)$ is a prior distribution of $\theta$. So, the interpretation here is that the smaller a Bayes factor is, the less likely it is that the parameter will take value $\theta$ having observed data $x$.

Also a p-value can be transformed into Bayes factor in a way of a \textit{calibrator}, which is that a non-decreasing and continuous function $f:(0,1) \rightarrow (0, + \infty)$ is a calibrator if and only if
$$
\int^1_0 \{ \frac{1}{f(p)} \} dp \leq 1
$$ For instance, a variant of calibrators is given by $f(p) := \frac{p^{1-\alpha}}{\alpha}$ for $\alpha \in (0,1)$

\section{Empirical Results}
The authors has tested the three machine learning algorithms combined with NCM using the dataset of the product of a Hihg Throughput Screening assay aimed at identifying chemical compounds that kill cells from a particular tumoral cell line. And the classification into Active vs. Inactive was carried out by applying a threshold on the estimated percentage of cells still alive after exposure to the chemical. And the chosen algorithms are 
\begin{itemize}
    \item Neural network: adapting conforma predictor to the output of the neuron in the output layer.
    \item SVM: NCM is $-y_i(x_i)$
    \item Random Forests: NCM chosen for RF was the fraction of trees that classified the test object as having the opposite label as the typothetical one
\end{itemize} I would like to refer the readers to the original paper regarding the detailed report of the empirical results.

\section{Conclusion}
 The study demonstrated on a real-world example that, despite their simplicity, these techniques can be of benefit, in particular with the Fisher method exhibiting a synergistic effect on the accuracy of ranking as in the case of the combination of NN and SVM. As for future work, since the tested algorithms only used the bare p-values, we can combine several CP, like \textit{Mixture of Experts models} by (Jacobs et al.,1991 \cite{jacobs1991adaptive})

\medskip
 
\printbibliography

\end{document}